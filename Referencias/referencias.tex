%\section{Referencias}

\begin{thebibliography}{X}
	
	\bibitem{Referencia1} \textsc{Sara Herrero Jaén} \textit{Formalización del concepto de salud a través de la lógica: impacto del lenguaje formal en las ciencias de la salud} \textit{http://scielo.isciii.es/scielo.php?script=sci\_arttext\&pid=S1988-348X2016000200006} 
	
	\bibitem{Referencia2} \textsc{Artículo Medicina Natural-Educación para la Salud} \textit{Qué es Enfermedad y Salud. Definiciones} \textit{https://es.scribd.com/document/156047411/Que-es-Enfermedad-y-Salud} 
	
	\bibitem{Referencia3} \textsc{PHFarma} \textit{¿Qué es un medicamento?} \textit{http://argentina.pmfarma.com/articulos/481--uso-racional-del-medicamento.html} 
	
	\bibitem{Referencia4} \textsc{Psicología y Mente} \textit{Tipos de medicamentos (según su uso y efectos secundarios)} \textit{https://psicologiaymente.com/salud/tipos-de-medicamentos}
	
	\bibitem{Referencia5} \textsc{CedimCat (Centre d'Informació de Medicaments de Catalunya)} \textit{CONCEPTO DE MEDICAMENTO} \textit{http://www.cedimcat.info/index.php?option=com\_content\&view=article\&id=209:que-es-un-medicamento\&catid=40\&Itemid=472\&lang=es} \textsc{July 2016}
	
	\bibitem{Referencia6} \textsc{La Comisión Nacional de Arbitraje Médico} \textit{GLOSARIO DE TÉRMINOS MÉDICO-JURÍDICOS} \textit{http://www.conamed.gob.mx/comisiones\_estatales/coesamed\_nayarit/
publicaciones/pdf/glosario.pdf}
	
	\bibitem{Referencia7} \textsc{Agencia Española de Medicamentos y Productos Sanitarios.} \textit{GUÍA DE PRESCRIPCIÓN TERAPÉUTICA} \textit{Barcelona: Pharma} \textsc{2016}
	
	\bibitem{Referencia8} \textsc{P.S. Raymundo} y \textsc{R.S. Octavio}, \textit{SEMINARIO: EL EJERCICIO ACTUAL DE LA MEDICINA}
	\textit{http://www.facmed.unam.mx/sms/seam2k1/2001/ponencia\_may\_2k1.htm}, EL PAPEL DE LA MEDICINA GENERAL EN EL SISTEMA NACIONAL DE SALUD
	
	\bibitem{Referencia9} \textsc{CommentCaMarche.net} \textit{CONCEPTO DE PACIENTE} \textit{https://salud.ccm.net/faq/15489-paciente-definicion}
	
	\bibitem{Referencia10} \textsc{eumed.net Enciclopedia virtual} \textit{CONCEPTO DE APLICACIÓN MÓVIL} \textit{http://www.eumed.net/libros-gratis/2016/1539/aplicacion.htm}
	
	\bibitem{Referencia11} \textsc{Abalit Technologies} \textit{¿Qué impacto social tienen las aplicaciones móviles?} \textit{https://www.abalit.org/blog/post/impacto-social-apps/es}
	
	\bibitem{Referencia12} \textsc{Marc Rubiño} \textit{Xamarin Forms uno para todos !!!} \textit{https://mrubino.net/2014/09/24/xamarin-forms-uno-para-todos/}
	
	\bibitem{Referencia13} \textsc{Rafael M} y \textsc{H.D. Sherali},
	\textit{http://www.milenio.com/ciencia-y-salud/mexico-4-10-abandonan-tratamiento-eli-lilly},  En México, 4 de cada 10 abandonan su tratamiento: Eli Lilly. 12.07.2017.
	
	\bibitem{Referencia14} \textsc{Organización Mundial de la Salud} \textit{Adherence to Long-Term Therapies. Evidence for Action} \textit{http://www.who.int/mediacentre/news/releases/2003/pr54/es/} \textsc{July 2003}
	
	\bibitem{Referencia15} \textsc{Derek Yach} \textit{ADHERENCE TO LONG-TERM THERAPIES-Evidence for action
} \textsc{January 2003}
	
	\bibitem{Referencia16} \textsc{Pablo Herrera Salinas . (2015)} \textsc{. ¿POR QUÉ NO SEGUIMOS LOS TRATAMIENTOS MÉDICOS E INDICACIONES DE SALUD?. 2018, de CERES DESARROLLO HUMANOY LA SECCIÓN EN ESPAÑOL DEL BRITISH GESTALT JOURNAL Sitio web:} \textit{https://www.britishgestaltjournal.com/features/2015/1/19/por-qu-no-seguimos-los-tratamientos-mdicos-e-indicaciones-de-salud}

	\bibitem{Referencia17} \textsc{Generacion Googleinstein} \textit{CONCEPTO DE DEPRECIACIÓN} \textit{http://financierosudl.blogspot.com/2009/04/concepto-de-depreciacion.html} \textsc{June 2018}
	
	\bibitem{Referencia18} \textsc{Nativos Digitales} \textit{CONOCIENDO LA METODOLOGÍA SCRUM: SUS BENEFICIOS Y CARACTERÍSTICAS} \textit{https://nativosdigitales.pe/blog/metodologia-scrum-beneficios-caracteristicas} \textsc{March 2017}
	
	\bibitem{Referencia19} \textsc{Proyectos Agiles.org} \textit{Qué es SCRUM} \textit{https://proyectosagiles.org/que-es-scrum/} 
	
	\bibitem{Referencia20} \textsc{El Androide Libre} \textit{¿Qué es Firebase? La mejorada plataforma de desarrollo de Google} \textit{https://elandroidelibre.elespanol.com/2016/05/firebase-plataforma-desarrollo-android-ios-web.html} 
	
	\bibitem{Referencia21} \textsc{Open Classrooms} \textit{En qué consiste el modelo en cascada} \textit{https://openclassrooms.com/en/courses/4309151-gestiona-tu-proyecto-de-desarrollo/4538221-en-que-consiste-el-modelo-en-cascada} 
	
%=====================================
	
		 
	
	 
\end{thebibliography}