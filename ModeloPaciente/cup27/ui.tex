\subsection{IUP27 Eliminar Contacto de Emergencia}
 
\subsubsection{Objetivo}

    Esta pantalla permite al actor eliminar sus contactos de emergencia.

\subsubsection{Diseño}

    En la figura \ref{IUP27} se muestra la pantalla ``Eliminar Contacto de Emergencia'', por medio de la cual se podrá eliminar sus contactos de emergencia.\\

    \IUfig[.3]{pantallas/eliminarContacto}{IUP27}{Eliminar Contacto de Emergencia}

%\subsubsection{Comandos}
%\begin{itemize}
%    \item \cdtButton{Aceptar}: Permite al actor ingresar al sistema, dirige a la pantalla de Bienvenida que se muestra en la figura \ref{fig:inicio}.
%    \item \cdtButton{Recuperar contraseña}: Permite al actor solicitar el envío de su contraseña a su correo, dirige a la pantalla \cdtIdRef{IUR 2}{Recuperar contraseña}.
%    \item \cdtButton{Inscribirse al programa}: Se utiliza solicitar la inscripción de una escuela para formar parte del programa, dirige a la pantalla \cdtIdRef{IUR 3}{Solicitar preinscripción}.
%\end{itemize}
%
%\subsubsection{Mensajes}
%
%\begin{description}
%    \item[\cdtIdRef{MSG5}{Falta un dato requerido para efectuar la operación solicitada}:] Se muestra en la pantalla \cdtIdRef{IUR 1}{Iniciar de sesión} cuando el actor omitió un dato marcado como requerido.
%    \item[\cdtIdRef{MSG22}{Nombre de usuario y/o contraseña incorrecto}:] Se muestra en la pantalla \cdtIdRef{IUR 1}{Iniciar de sesión} indicando que el nombre de usuario y/o contraseña son incorrectos.
%\end{description}
