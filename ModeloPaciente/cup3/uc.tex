\begin{UseCase}{CUR 3}{Registro de cuenta}
    {

    	
    	El registro de la cuenta del usuario consiste en que el usuario debe de ingresar todos los datos correspondientes a su perfil y seleccionar los roles a los cuales estará asociado, para que se le puedan asignar las acciones a las que podra acceder.
    
    	
    }
    \UCitem{Versión}{1.0}
    \UCccsection{Paciente}
    \UCitem{Autor}{Victor Arquimedes Estrada Machuca}
    \UCccitem{Evaluador}{Daniel Josue Fuentes Hernandez}
    \UCitem{Operación}{Crear usuario}
    \UCccitem{Prioridad}{Alta}
    \UCccitem{Complejidad}{Baja}
    \UCccitem{Volatilidad}{Baja}
    \UCccitem{Madurez}{Alta}
    \UCitem{Estatus}{Terminado}
    \UCitem{Fecha del último estatus}{05 de Octubre del 2018}

%% Copie y pegue este bloque tantas veces como revisiones tenga el caso de uso.
%% Esta sección la debe llenar solo el Revisor
% %--------------------------------------------------------
% 	\UCccsection{Revisión Versión XX} % Anote la versión que se revisó.
% 	% FECHA: Anote la fecha en que se terminó la revisión.
% 	\UCccitem{Fecha}{Fecha en que se termino la revisión} 
% 	% EVALUADOR: Coloque el nombre completo de quien realizó la revisión.
% 	\UCccitem{Evaluador}{Nombre de quien revisó}
% 	% RESULTADO: Coloque la palabra que mas se apegue al tipo de acción que el analista debe realizar.
% 	\UCccitem{Resultado}{Corregir, Desechar, Rehacer todo, terminar.}
% 	% OBSERVACIONES: Liste los cambios que debe realizar el Analista.
% 	\UCccitem{Observaciones}{
% 		\begin{UClist}
% 			% PC: Petición de Cambio, describa el trabajo a realizar, si es posible indique la causa de la PC. Opcionalmente especifique la fecha en que considera razonable que se deba terminar la PC. No olvide que la numeración no se debe reiniciar en una segunda o tercera revisión.
% 			\RCitem{PC1}{\TODO{Descripción del pendiente}}{Fecha de entrega}
% 			\RCitem{PC2}{\TODO{Descripción del pendiente}}{Fecha de entrega}
% 			\RCitem{PC3}{\TODO{Descripción del pendiente}}{Fecha de entrega}
% 		\end{UClist}		
% 	}
% %--------------------------------------------------------
% %--------------------------------------------------------
% 	\UCccsection{Revisión Versión 0.3} % Anote la versión que se revisó.
% 	% FECHA: Anote la fecha en que se terminó la revisión.
% 	\UCccitem{Fecha}{04/11/2014} 
% 	% EVALUADOR: Coloque el nombre completo de quien realizó la revisión.
% 	\UCccitem{Evaluador}{Natalia Giselle Hernández Sánchez}
% 	% RESULTADO: Coloque la palabra que mas se apegue al tipo de acción que el analista debe realizar.
% 	\UCccitem{Resultado}{Corregir}
% 	% OBSERVACIONES: Liste los cambios que debe realizar el Analista.
% 	\UCccitem{Observaciones}{
% 		\begin{UClist}
% 			% PC: Petición de Cambio, describa el trabajo a realizar, si es posible indique la causa de la PC. Opcionalmente especifique la fecha en que considera razonable que se deba terminar la PC. No olvide que la numeración no se debe reiniciar en una segunda o tercera revisión.
% 			\RCitem{PC2}{\DONE{Cambiar en la sección de entradas el nombre del atributo ``Nombre de usuario'' y la redirección}}{04/11}
% 			\RCitem{PC3}{\DONE{La liga de ``Contraseña'' está rota}}{04/11}
% 			\RCitem{PC4}{\DONE{Hay oraciones que no tienen punto al final, como las precondiciones}}{04/11}
% 			\RCitem{PC5}{\DONE{Quitar la palabra ``interna'' de las viñetasde los menús en la sección de postcondiciones}}{04/11}
% 			\RCitem{PC6}{\DONE{En las postcondiciones dice ``perfl''}}{04/11}
% 			\RCitem{PC7}{\DONE{Falta punto en el primer paso de la TP}}{05/11}
% 			\RCitem{PC8}{\DONE{El paso 6 de la trayectoria principal no debería de estar debido a que el nombre de usuario no siempre es la cct}}{05/11}
% 			\RCitem{PC9}{\DONE{El paso 5 de la TP podría ser general como ``Verifica que los datos ingresados sean correctos según la regla de negocio...'' para que se verifique la contraseña, en todos los campos se verifica la longitud}}{05/11}
% 			\RCitem{PC10}{\DONE{Hace falta mencionar el mensaje de longitud}}{05/11}
% 			\RCitem{PC11}{\DONE{Falta mencionar el mensaje 22 en la sección de errores}}{05/11}
% 			\RCitem{PC12}{\DONE{Falta el mensaje de que se ha excedido la longitud}}{05/11}
% 			\RCitem{PC13}{\DONE{Falta el mensaje de formato incorrecto}}{05/11}
% 			\RCitem{PC14}{\DONE{En la trayectoria alternativa C el campo no debería de indicarse}}{05/11}
% 			\RCitem{PC15}{\DONE{En el paso 8 indicar que se trata de una pantalla en lugar de decir ``figura''}}{05/11}
% 			\RCitem{PC16}{\DONE{¿Se hablará de cómo cerrar la sesión?, en el sistema pasado se documento ``cerrar sesón'' como parte de la trayectoria principal del caso de uso ``Ingresar al sistema'' }}{05/11}
% 			\RCitem{PC17}{\DONE{En el segunto punto de extensión dice ``insripción''}}{05/11}
% 			
% 		\end{UClist}		
% 	}
%--------------------------------------------------------
	\UCccsection{Revisión Versión 0.3} % Anote la versión que se revisó.
	% FECHA: Anote la fecha en que se terminó la revisión.
	\UCccitem{Fecha}{11-11-14} 
	% EVALUADOR: Coloque el nombre completo de quien realizó la revisión.
	\UCccitem{Evaluador}{Natalia Giselle Hernández Sánchez}
	% RESULTADO: Coloque la palabra que mas se apegue al tipo de acción que el analista debe realizar.
	\UCccitem{Resultado}{Corregir}
	% OBSERVACIONES: Liste los cambios que debe realizar el Analista.
	\UCccitem{Observaciones}{
		\begin{UClist}
			% PC: Petición de Cambio, describa el trabajo a realizar, si es posible indique la causa de la PC. Opcionalmente especifique la fecha en que considera razonable que se deba terminar la PC. No olvide que la numeración no se debe reiniciar en una segunda o tercera revisión.
			\RCitem{PC1}{\DONE{Agregar a precondiciones el estado de la cuenta}}{Fecha de entrega}
			\RCitem{PC2}{\DONE{Agregar el paso de la trayectoria de validación del estado de la cuenta}}{Fecha de entrega}
			\RCitem{PC3}{\DONE{Agregar el mensaje de cuenta no activada a la sección de errores}}{Fecha de entrega}
			\RCitem{PC4}{\DONE{Verificar las ligas a los estados}}{Fecha de entrega}
			
		\end{UClist}		
	}
%--------------------------------------------------------

	\UCsection{Atributos}
	\UCitem{Actor}{\textbf{Paciente}}
	\UCitem{Propósito}{Crear cuenta.}
	\UCitem{Entradas}{
        \begin{UClist} 
           \UCli \textbf{Nombre}: \ioEscribir.
           \UCli \textbf{Apellido paterno}: \ioEscribir.
           \UCli \textbf{Apellido materno}: \ioEscribir.
           \UCli \textbf{Fecha de nacimiento}.
           \UCli \textbf{Numero de celular}
           \UCli \textbf{Correo electronico}
           \UCli \textbf{Contraseña}
           \UCli \textbf{Rol del usuario}: Se selecciona de checkbox
           \UCli \textbf{Numero de Cédula profesional}: \ioEscribir.
        \end{UClist}}
	\UCitem{Salidas}{Ninguna.}
	\UCitem{Precondiciones}{
		Ninguna.
		}
	\UCitem{Postcondiciones}{
	    {\bf Interna:} El actor podrá ingresar a la aplicación.	
	}
    \UCitem{Reglas de negocio}{

            \cdtIdRef{RN1}{Información correcta}: Verifica que la información introducida sea correcta.
            \cdtIdRef{RN3}{Contraseña invalida}: Verifica que la contraseña escrita cuente con la estructura correcta.
    }
	\UCitem{Errores}{
	    \begin{UClist}
	    	
		\UCli \cdtIdRef{MSG2}{Falta un dato requerido para efectuar la operación solicitada}: Se muestra en la pantalla \textbf{Crear cuenta} cuando el actor omitió un dato marcado como requerido.
		
	    \end{UClist}
	}
	\UCitem{Tipo}{Primario.}
	\UCitem{Fuente}{
	}
 \end{UseCase}

 \begin{UCtrayectoria}
 	
	 \UCpaso [\UCactor] Presiona el botón \cdtButton{Crear cuenta} de la pantalla \textbf{Iniciar sesión}.
	 \UCpaso Muestra la pantalla \textbf{Creación de cuenta}.
	 \UCpaso [\UCactor] Ingresa Nombre.
	 \UCpaso [\UCactor] Ingresa Apellido paterno.
	 \UCpaso [\UCactor] Ingresa Apellido materno.
	 \UCpaso [\UCactor] Ingresa Fecha de nacimiento.
	 \UCpaso [\UCactor] Presiona el boton \cdtButton{Obtener numero} \refTray{A}
	 \UCpaso Muestra el mensaje \textbf{MSG5{Obtener datos}}.
	 \UCpaso [\UCactor] Presiona el botón \cdtButton{Aceptar} en el mensaje.
	 \UCpaso Obtiene el numero celular del celular del usuario
	 \UCpaso La casilla \textbf{Celular} se rellena con la información obtenida del paso anterior.
	 \UCpaso [\UCactor] Ingresa su contraseña.\label{cur5:Celular}
	 \UCpaso [\UCactor] Selecciona los roles a los que estará asociado.\refTray{B}
	 \UCpaso [\UCactor] Presiona el botón \cdtButton{Siguiente}. \label{cur3:Doctor}
	 \UCpaso Verifica que los datos ingresados por el usuario sean correctos como lo indica la regla de negocio \cdtIdRef{RN1}{Información correcta}. \refTray{C}
	 \UCpaso Verifica que la contraseña sea correcta como lo indica la regla de negocio \cdtIdRef{RN3}{Contraseña}. \refTray{D}.
	 \UCpaso Muestra la pantalla \textbf{Creación de cuenta fotografia}.
	 \UCpaso [\UCactor] Da click en el icono \textbf{Tomar foto}. \refTray{E}
	 \UCpaso Muestra el mensaje \cdtIdRef{MSG5}{Obtener datos} en la pantalla \textbf{Creación de cuenta fotografia}.
	 \UCpaso [\UCactor] Presiona el boton \cdtButton{Aceptar} del mensaje.
	 \UCpaso Abre la camara del celular.
	 \UCpaso Obtiene la fotografia tomada.
	 \UCpaso Muesta el mensaje \cdtIdRef{MSG7}{Obtener otra foto} en la pantalla \textbf{Creación de cuenta fotografia}.
	 \UCpaso [\UCactor] Presiona el botón \cdtButton{No} del mensaje.
	 \UCpaso [\UCactor] Presiona el botón \cdtButton{Crear Cuenta} de la pantalla.\label{cur3:foto}
	 \UCpaso Muestra el mensaje \cdtIdRef{MSG1}{Operación Exitosa} en la pantalla \textbf{Iniciar sesión}.
	 

 \end{UCtrayectoria}

 \begin{UCtrayectoriaA}{A}{El actor no desea que la aplicación se comunique con su sistema operativo.}
 	\UCpaso [\UCactor] Ingresa su numero celular.
 	\UCpaso Continúa en el paso \ref{cur5:Celular} de la trayectoria principal.
    
 \end{UCtrayectoriaA}
 
 \begin{UCtrayectoriaA}{B}{El actor tendrá un rol que requiere un dato extra.}
 	\UCpaso [\UCactor] Selecciona la opción Doctor de la lista de checkbox.
 	\UCpaso Muestra la casilla \textbf{Cédula profesional} en la pantalla \textbf{Creación de cuenta}.
 	\UCpaso [\UCactor] Ingresa su cédula profesional.
 	\UCpaso Continúa con el paso \ref{cur3:Doctor} de la trayectoria principal.
 	
   
 \end{UCtrayectoriaA}
 
\begin{UCtrayectoriaA}{C}{El actor no ingresó alguno de los datos requeridos.}
	\UCpaso[\UCsist] Muestra el mensaje \cdtIdRef{MSG2}{Falta un dato requerido para efectuar la operación solicitada} en la pantalla \textbf{Creación de cuenta}
	\UCpaso[] Continúa en el paso donde no haya llenado el campo de la trayectoria principal.
\end{UCtrayectoriaA}

\begin{UCtrayectoriaA}{D}{El actor no ingreso la contraseña de forma correcta.}
	\UCpaso[\UCsist] Muestra el mensaje \cdtIdRef{MSG6}{Faltan datos en la contraseña} en la pantalla \textbf{Creación de cuenta}
	\UCpaso[] Continúa en el paso \ref{cur5:Celular} de la trayectoria principal.
\end{UCtrayectoriaA}

\begin{UCtrayectoriaA}{E}{El actor no quiere tomar una foto.}
	\UCpaso [\UCactor] Da click sobre el icono \textbf{Seleccionar de fototeca}.
	\UCpaso Muestra el mensaje \cdtIdRef{MSG5}{Obtener datos} en la pantalla \textbf{Creación de cuenta fotografia} la primera vez que esto sucede.
	\UCpaso [\UCactor] Presiona el boton \cdtButton{Aceptar} del mensaje.
	\UCpaso Muestra las fotos del celular.
	\UCpaso Obtiene la foto seleccionada.
	\UCpaso Continúa en el paso \ref{cur3:foto} de la trayectoria principal.
	
\end{UCtrayectoriaA}
 
\subsection{Puntos de extensión}

 

