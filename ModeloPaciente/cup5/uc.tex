\begin{UseCase}{CUR 5}{Editar Datos Personales}
    {

    	Los datos al pasar el tiempo son flexibles al cambio, esto quiere decir que un usuario que inicialmente solo tenia un rol, ahora puede tener uno o dos mas.
    	De igual manera el peso, la altura, el correo electrónico e incluso la contraseña se pueden ver afectadas por el tiempo y es necesario actualizar estos datos.
    	
    }
    \UCitem{Versión}{1.0}
    \UCccsection{Paciente}
    \UCitem{Autor}{Victor Arquimedes Estrada Machuca}
    \UCccitem{Evaluador}{Daniel Josue Fuentes Hernandez}
    \UCitem{Operación}{Editar datos personale}
    \UCccitem{Prioridad}{Media}
    \UCccitem{Complejidad}{Baja}
    \UCccitem{Volatilidad}{Baja}
    \UCccitem{Madurez}{Alta}
    \UCitem{Estatus}{Terminado}
    \UCitem{Fecha del último estatus}{05 de Octubre del 2018}



%--------------------------------------------------------

	\UCsection{Atributos}
	\UCitem{Actor}{\textbf{Paciente}}
	\UCitem{Propósito}{Editar datos.}
	\UCitem{Entradas}{
        \begin{UClist} 
           \UCli Peso: \ioEscribir.
           \UCli Altura: \ioEscribir.
           \UCli Roles del usuario: Se selecciona de checkbox.
           \UCli Contraseña: \ioEscribir.
           \UCli Foto de perfil: La selecciona de la fototeca ó realiza una fotografía.
        \end{UClist}
    }
	
	\UCitem{Salidas}{
		\begin{UClist}
			\UCli Nombre del usuario
			\UCli Rol utilizado en ese momento.
			\UCli Foto de perfil.
			\UCli Categoría a la que pertenece.
			\UCli Fecha de nacimiento.
			\UCli Sexo.
			\UCli Peso.
			\UCli Altura
			\UCli Correo electrónico.
			\UCli Contraseña.
			
		\end{UClist}
	}
	\UCitem{Precondiciones}{
		Ninguna.
		}
	\UCitem{Postcondiciones}{
	   {\bf Interna:} Los datos serán actualizados.
	}
    \UCitem{Reglas de negocio}{

            \cdtIdRef{RN1}{Información correcta}: Verifica que la información introducida sea correcta.
            \cdtIdRef{RN4}{Roles existentes}: Verifica que el usuario cuente con al menos un rol.
            \cdtIdRef{RN3}{Contraseña invalida}: Verifica que la contraseña escrita cuente con la estructura correcta.
            \cdtIdRef{RN5}{Cédula profesional}: Verifica que la cédula profesional exista y sea real.
    }
	\UCitem{Errores}{
	    \begin{UClist}
	    \UCli \cdtIdRef{MSG1}{Operación Exitosa}: Se muestra en la pantalla \textbf{Consultar datos} cuando se actualizaron de forma correcta los datos.	
		\UCli \cdtIdRef{MSG2}{Falta un dato requerido para efectuar la operación solicitada}: Se muestra en la pantalla \textbf{Iniciar sesión} cuando el actor omitió un dato marcado como requerido.
		\UCli \cdtIdRef{MSG6}{Faltan datos en la contraseña}
		\UCli \cdtIdRef{MSG10}{Correo electrónico nuevo}
		\UCli \cdtIdRef{MSG11}{No hay rol}
		\UCli \cdtIdRef{MSG12}{Cédula profesional no valida} 
		
	    \end{UClist}
	}
	\UCitem{Tipo}{Primario.}
	\UCitem{Fuente}{
	}
 \end{UseCase}

 \begin{UCtrayectoria}
 	
 	\UCpaso [\UCactor] Da click sobre el icono de \textbf{Editar} en la pantalla \textbf{Datos personales}.
 	\UCpaso Deshabilita el icono \textbf{Contactos de Emergencia}.
 	\UCpaso Se habilitan las casillas que son modificables. 	
 	\UCpaso [\UCactor] Da click en el icono \textbf{Tomar foto}. \refTray{A}
 	\UCpaso Muestra el mensaje \cdtIdRef{MSG5}{Obtener datos} en la pantalla \textbf{Creación de cuenta fotografia}.
 	\UCpaso Abre la cámara del celular.\label{cur:otrafoto}
 	\UCpaso Obtiene la fotografia tomada.
 	\UCpaso Muestra el mensaje \cdtIdRef{MSG7}{Obtener otra foto} en la pantalla \textbf{Creación de cuenta fotografia}.
 	\UCpaso [\UCactor] Presiona el botón \cdtButton{No} del mensaje.\refTray{B}
 	\UCpaso [\UCactor] Ingresa su nuevo peso.\label{cur5:foto}
 	\UCpaso [\UCactor] Ingresa su nueva altura.
 	\UCpaso [\UCactor] Selecciona un checkbox para habilitar un rol.  \refTray{C} \label{roles}
 	\UCpaso [\UCactor] Ingresa su nuevo correo electrónico. \refTray{D} \label{correo}
 	\UCpaso [\UCactor] Presiona el botón \cdtButton{Guardar cambios}.\refTray{E}
 	\UCpaso Muestra el mensaje \cdtIdRef{MSG10}{Correo electronico nuevo}.
 	\UCpaso [\UCactor] Presiona el botón \cdtButton{Aceptar}.
 	\UCpaso Verifica que los datos ingresados por el usuario sean correctos como lo indica la regla de negocio \cdtIdRef{RN1}{Información correcta}. \refTray{F}
 	\UCpaso Verifica que el usuario tenga por lo menos un rol como lo indica la regla de negocio \cdtIdRef{RN4}{Roles existentes}. \refTray{G}.
 	\UCpaso Envía una confirmación de cambio de correo electrónico al nuevo correo electrónico ingresado en el paso \ref{correo}
 	\UCpaso Actualiza los datos del usuario.
 	\UCpaso Muestra el mensaje \cdtIdRef{MSG1}{Operación Exitosa} en la pantalla \textbf{Consultar datos personales}.
 	
 	 
 	

 \end{UCtrayectoria}

 \begin{UCtrayectoriaA}{A}{El actor no quiere tomar una foto.}
 	\UCpaso [\UCactor] Da click sobre el icono \textbf{Seleccionar de fototeca}.
 	\UCpaso Muestra el mensaje \cdtIdRef{MSG5}{Obtener datos} en la pantalla \textbf{Creación de cuenta fotografia} la primera vez que esto sucede.
 	\UCpaso [\UCactor] Presiona el boton \cdtButton{Aceptar} del mensaje.
 	\UCpaso Muestra las fotos del celular.
 	\UCpaso Obtiene la foto seleccionada.
 	\UCpaso Continúa en el paso \ref{cur5:foto} de la trayectoria principal.
 	
 \end{UCtrayectoriaA}

\begin{UCtrayectoriaA}{B}{El actor desea tomar otra foto}
	\UCpaso [\UCactor] Presiona el botón \cdtButton{Si} del mensaje \cdtIdRef{MSG7}{Obtener otra foto}.
	\UCpaso Continua en el paso \ref{cur:otrafoto}
\end{UCtrayectoriaA}


\begin{UCtrayectoriaA}{C}{El actor selecciona un rol que necesita un dato extra}
	
	\UCpaso [\UCactor] Selecciona el checkbox del rol \textbf{Doctor}.
	\UCpaso Habilita la casilla de cédula profesional.
	\UCpaso [\UCactor] Ingresa su cédula profesional. \label{Cedula}
	\UCpaso [\UCactor] Presiona el botón \cdtButton{Guardar cambios}.\refTray{E}
	\UCpaso Verifica que los datos ingresados por el usuario sean correctos como lo indica la regla de negocio \cdtIdRef{RN1}{Información correcta}. \refTray{F}
	\UCpaso Verifica que el usuario tenga por lo menos un rol como lo indica la regla de negocio \cdtIdRef{RN4}{Roles existentes}. \refTray{G}.
	\UCpaso Verifica que la cédula profesional ingresada este como lo indica la regla de negocio \cdtIdRef{RN4}{Roles existentes}. \refTray{H}.
	\UCpaso Actualiza los datos del usuario.
	\UCpaso Muestra el mensaje \cdtIdRef{MSG1}{Operación Exitosa} en la pantalla \textbf{Consultar datos personales}.
\end{UCtrayectoriaA}


\begin{UCtrayectoriaA}{D}{El actor no desea cambiar de correo electrónico}
	
	\UCpaso [\UCactor] Presiona el botón \cdtButton{Guardar cambios}.\refTray{E}
	\UCpaso Verifica que los datos ingresados por el usuario sean correctos como lo indica la regla de negocio \cdtIdRef{RN1}{Información correcta}. \refTray{F}
	\UCpaso Verifica que el usuario tenga por lo menos un rol como lo indica la regla de negocio \cdtIdRef{RN4}{Roles existentes}. \refTray{G}.
	\UCpaso Actualiza los datos del usuario.
	\UCpaso Muestra el mensaje \cdtIdRef{MSG1}{Operación Exitosa} en la pantalla \textbf{Consultar datos personales}.
\end{UCtrayectoriaA}



\begin{UCtrayectoriaA}{E}{El actor no desea modificar ningún dato}
		
	\UCpaso [\UCactor] Presiona el botón \cdtButton{Cancelar}.
	\UCpaso Muestra la pantalla \textbf{Consultar Datos personales}.

\end{UCtrayectoriaA}

\begin{UCtrayectoriaA}{F}{El actor no ingresó alguno de los datos requeridos.}
	\UCpaso[\UCsist] Muestra el mensaje \cdtIdRef{MSG2}{Falta un dato requerido para efectuar la operación solicitada} en la pantalla \textbf{Editar datos personales}
	\UCpaso[] Continúa en el paso donde no haya llenado el campo de la trayectoria principal.
\end{UCtrayectoriaA}

\begin{UCtrayectoriaA}{G}{El actor no selecciono ningún rol.}
	
	\UCpaso[\UCsist] Muestra el mensaje\cdtIdRef{MSG11}{No hay rol} en la pantalla \textbf{Editar datos personales}
	\UCpaso[] Continúa en el paso \ref{roles}.
	
\end{UCtrayectoriaA}


\begin{UCtrayectoriaA}{H}{El actor ingreso una cédula no valida.}
	
	\UCpaso[\UCsist] Muestra el mensaje\cdtIdRef{MSG12}{Cédula profesional no valida} en la pantalla \textbf{Editar datos personales}
	\UCpaso[] Continúa en el paso \ref{Cedula}.
	
\end{UCtrayectoriaA}
 
\subsection{Puntos de extensión}



 

