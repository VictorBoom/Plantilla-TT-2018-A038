\begin{UseCase}{CUP10}{Agregar Tratamiento}
    {

    	Cuando un paciente se enferma y recurre a visitar al doctor, éste le dará un nuevo tratamiento, con nuevas instrucciones y medicamentos, entonces de esta manera el usuario podrá agregar un tratamiento nuevo en donde establecerá cuantos medicamentos tiene que tomar, cuándo se expidió el tratamiento y quién es el doctor encargado de este.
    	
    	
    }
    \UCitem{Versión}{1.0}
    \UCccsection{Paciente}
    \UCitem{Autor}{Victor Arquimedes Estrada Machuca}
    \UCccitem{Evaluador}{Daniel Josue Fuentes Hernández}
    \UCitem{Operación}{Agregar Tratamiento}
    \UCccitem{Prioridad}{Alta}
    \UCccitem{Complejidad}{Baja}
    \UCccitem{Volatilidad}{Baja}
    \UCccitem{Madurez}{Alta}
    \UCitem{Estatus}{Terminado}
    \UCitem{Fecha del último estatus}{05 de Octubre del 2018}


	\UCsection{Atributos}
	\UCitem{Actor}{\textbf{Paciente}}
	\UCitem{Propósito}{Agregar Tratamiento}
	\UCitem{Entradas}{
        \begin{UClist} 
           \UCli Número de medicamentos.
           \UCli Fecha de Expedición del Tratamiento.
           \UCli Doctor encargado del tratamiento
        \end{UClist}
    }
	\UCitem{Salidas}{
		\begin{UClist}
			\UCli Nombre del Tratamiento.
			\UCli Nombre del Doctor.
			\UCli Cédula del Doctor.
		\end{UClist}
		}
	\UCitem{Precondiciones}{
		{\bf Interna:} El actor debe estár registrado en el sistema.
		}
	\UCitem{Postcondiciones}{
	    {\bf Interna:} El actor tendrá un doctor.	
	}
    \UCitem{Reglas de negocio}{

            \cdtIdRef{RN6}{Doctor encargado}: Verifica que haya un doctor encargado del tratamiento.
    }
	\UCitem{Mensajes}{
	    \begin{UClist}
	    \UCli \cdtIdRef{MSG1}{Operación Exitosa}: Se muestra en la pantalla \textbf{Iniciar sesión} cuando se inició la sesión de manera exitosa.	
		\UCli \cdtIdRef{MSG2}{Falta un dato requerido para efectuar la operación solicitada}: Se muestra en la pantalla \textbf{Iniciar sesión} cuando el actor omitió un dato marcado como requerido.
		\UCli \cdtIdRef{MSG3}{Correo electrónico y/o contraseña incorrecto}: Se muestra en la pantalla \textbf{Iniciar sesión} indicando que el correo electrónico y/o contraseña son incorrectos.
		\UCli \cdtIdRef{MSG13}{No existe el doctor}
		
		
	    \end{UClist}
	}
	\UCitem{Tipo}{Primario.}
	\UCitem{Fuente}{
	}
 \end{UseCase}

 \begin{UCtrayectoria}
 	
 	\UCpaso [\UCactor] Presiona el ícono \textbf{Agregar Tratamiento} de la pantalla \textbf{Tratamientos}.
 	\UCpaso Genera el nombre del nuevo tratamiento, siguiendo el orden numérico de los nombres creados hasta el momento.
 	\UCpaso Muestra la pantalla \cdtIdRef{Agregar Nuevo Tratamiento}.
 	\UCpaso [\UCactor] Selecciona cuántos medicamentos tiene el tratamiento.
 	\UCpaso [\UCactor] Selecciona la fecha de expedición del tratamiento.
 	\UCpaso [\UCactor] Ingresa la cédula profesional o el nombre del doctor.
 	\UCpaso [\UCactor] Selecciona el doctor que expidió el tratamiento.\refTray{A}
 	\UCpaso Muestra el nombre del doctor.
 	\UCpaso Muestra la cédula del doctor.
 	\UCpaso [\UCactor] Presiona el botón \cdtButton{Agregar Tratamiento}. \label{cur9:doc}
 	\UCpaso Verifica que haya al menos un medicamento en el tratamiento.
 	\UCpaso Muestra la pantalla \textbf{Consultar Tratamientos}.
 	
 	
    
 \end{UCtrayectoria}

\begin{UCtrayectoriaA}{A}{No se encuentra el doctor.}
	\UCpaso Muestra el mensaje \cdtIdRef{MSG13}{No existe el doctor}
	\UCpaso Extiende al caso de uso \cdtIdRef{CUR11}{Agregar Doctor} 
	\UCpaso Continúa con el paso \ref{cur9:doc} de la trayectoria principal.
\end{UCtrayectoriaA} 
 



 
\subsection{Puntos de extensión}

\UCExtensionPoint
{El actor requiere agregar un doctor}
{ Paso \ref{cur9:doc} de la trayectoria principal}
{\cdtIdRef{CUR11}{Agregar Doctor}}



