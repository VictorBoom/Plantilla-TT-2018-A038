\subsection{IUP9 Editar Tratamiento}
 
\subsubsection{Objetivo}

    Esta pantalla permite al actor editar los tratamientos médicos que estén en estado Incompleto.

\subsubsection{Diseño}

    En la figura \ref{IUP9} se muestra la pantalla ``Editar Tratamiento'', por medio de la cual se podrá editar los tratamientos que estén en estado Incompleto. \\
	\newpage
    \IUfig[.3]{pantallas/editarTratamiento}{IUP9}{Editar Tratamiento}

\subsubsection{Comandos}
\begin{itemize}
    \item \cdtButton{Agregar Medicamento}: Permite al actor agregar un nuevo medicamento a un tratamiento en estado incompleto.
    
\end{itemize}

%\subsubsection{Mensajes}
%
%\begin{description}
%    \item[\cdtIdRef{MSG5}{Falta un dato requerido para efectuar la operación solicitada}:] Se muestra en la pantalla \cdtIdRef{IUR 1}{Iniciar de sesión} cuando el actor omitió un dato marcado como requerido.
%    \item[\cdtIdRef{MSG22}{Nombre de usuario y/o contraseña incorrecto}:] Se muestra en la pantalla \cdtIdRef{IUR 1}{Iniciar de sesión} indicando que el nombre de usuario y/o contraseña son incorrectos.
%\end{description}
