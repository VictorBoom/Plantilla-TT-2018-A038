\subsection{IUP23 Eliminar Auxiliar}
 
\subsubsection{Objetivo}

    Esta pantalla permite al actor eliminar a un auxiliar.

\subsubsection{Diseño}

    En la figura \ref{IUP23} se muestra la pantalla ``Eliminar Auxiliar'', por medio de la cual se podrá eliminar los auxiliares que ya no sean necesarios. \\

    \IUfig[.3]{pantallas/eliminarAuxiliar}{IUP23}{Eliminar Auxiliar}

\subsubsection{Comandos}
\begin{itemize}
    \item \cdtButton{Eliminar}: Permite al actor eliminar un auxiliar.
    \item \cdtButton{Cancelar}: Permite al actor cancelar la operación.
    
\end{itemize}

%\subsubsection{Mensajes}
%
%\begin{description}
%    \item[\cdtIdRef{MSG5}{Falta un dato requerido para efectuar la operación solicitada}:] Se muestra en la pantalla \cdtIdRef{IUR 1}{Iniciar de sesión} cuando el actor omitió un dato marcado como requerido.
%    \item[\cdtIdRef{MSG22}{Nombre de usuario y/o contraseña incorrecto}:] Se muestra en la pantalla \cdtIdRef{IUR 1}{Iniciar de sesión} indicando que el nombre de usuario y/o contraseña son incorrectos.
%\end{description}
