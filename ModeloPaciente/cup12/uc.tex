\begin{UseCase}{CUR 11}{Agregar Doctor}
    {

    	
    	Un doctor es quien esta acargo de un tratamiento, es por eso que es importante agregar al o a los doctores que estarán llevando los tratamientos médicos de un paciente.
   	
    }
    \UCitem{Versión}{1.0}
    \UCccsection{Paciente}
    \UCitem{Autor}{Victor Arquimedes Estrada Machuca}
    \UCccitem{Evaluador}{Daniel Josue Fuentes Hernandez}
    \UCitem{Operación}{Agregar Doctor}
    \UCccitem{Prioridad}{Alta}
    \UCccitem{Complejidad}{Baja}
    \UCccitem{Volatilidad}{Baja}
    \UCccitem{Madurez}{Alta}
    \UCitem{Estatus}{Terminado}
    \UCitem{Fecha del último estatus}{05 de Octubre del 2018}

%% Copie y pegue este bloque tantas veces como revisiones tenga el caso de uso.
%% Esta sección la debe llenar solo el Revisor
% %--------------------------------------------------------
% 	\UCccsection{Revisión Versión XX} % Anote la versión que se revisó.
% 	% FECHA: Anote la fecha en que se terminó la revisión.
% 	\UCccitem{Fecha}{Fecha en que se termino la revisión} 
% 	% EVALUADOR: Coloque el nombre completo de quien realizó la revisión.
% 	\UCccitem{Evaluador}{Nombre de quien revisó}
% 	% RESULTADO: Coloque la palabra que mas se apegue al tipo de acción que el analista debe realizar.
% 	\UCccitem{Resultado}{Corregir, Desechar, Rehacer todo, terminar.}
% 	% OBSERVACIONES: Liste los cambios que debe realizar el Analista.
% 	\UCccitem{Observaciones}{
% 		\begin{UClist}
% 			% PC: Petición de Cambio, describa el trabajo a realizar, si es posible indique la causa de la PC. Opcionalmente especifique la fecha en que considera razonable que se deba terminar la PC. No olvide que la numeración no se debe reiniciar en una segunda o tercera revisión.
% 			\RCitem{PC1}{\TODO{Descripción del pendiente}}{Fecha de entrega}
% 			\RCitem{PC2}{\TODO{Descripción del pendiente}}{Fecha de entrega}
% 			\RCitem{PC3}{\TODO{Descripción del pendiente}}{Fecha de entrega}
% 		\end{UClist}		
% 	}
% %--------------------------------------------------------
% %--------------------------------------------------------
% 	\UCccsection{Revisión Versión 0.3} % Anote la versión que se revisó.
% 	% FECHA: Anote la fecha en que se terminó la revisión.
% 	\UCccitem{Fecha}{04/11/2014} 
% 	% EVALUADOR: Coloque el nombre completo de quien realizó la revisión.
% 	\UCccitem{Evaluador}{Natalia Giselle Hernández Sánchez}
% 	% RESULTADO: Coloque la palabra que mas se apegue al tipo de acción que el analista debe realizar.
% 	\UCccitem{Resultado}{Corregir}
% 	% OBSERVACIONES: Liste los cambios que debe realizar el Analista.
% 	\UCccitem{Observaciones}{
% 		\begin{UClist}
% 			% PC: Petición de Cambio, describa el trabajo a realizar, si es posible indique la causa de la PC. Opcionalmente especifique la fecha en que considera razonable que se deba terminar la PC. No olvide que la numeración no se debe reiniciar en una segunda o tercera revisión.
% 			\RCitem{PC2}{\DONE{Cambiar en la sección de entradas el nombre del atributo ``Nombre de usuario'' y la redirección}}{04/11}
% 			\RCitem{PC3}{\DONE{La liga de ``Contraseña'' está rota}}{04/11}
% 			\RCitem{PC4}{\DONE{Hay oraciones que no tienen punto al final, como las precondiciones}}{04/11}
% 			\RCitem{PC5}{\DONE{Quitar la palabra ``interna'' de las viñetasde los menús en la sección de postcondiciones}}{04/11}
% 			\RCitem{PC6}{\DONE{En las postcondiciones dice ``perfl''}}{04/11}
% 			\RCitem{PC7}{\DONE{Falta punto en el primer paso de la TP}}{05/11}
% 			\RCitem{PC8}{\DONE{El paso 6 de la trayectoria principal no debería de estar debido a que el nombre de usuario no siempre es la cct}}{05/11}
% 			\RCitem{PC9}{\DONE{El paso 5 de la TP podría ser general como ``Verifica que los datos ingresados sean correctos según la regla de negocio...'' para que se verifique la contraseña, en todos los campos se verifica la longitud}}{05/11}
% 			\RCitem{PC10}{\DONE{Hace falta mencionar el mensaje de longitud}}{05/11}
% 			\RCitem{PC11}{\DONE{Falta mencionar el mensaje 22 en la sección de errores}}{05/11}
% 			\RCitem{PC12}{\DONE{Falta el mensaje de que se ha excedido la longitud}}{05/11}
% 			\RCitem{PC13}{\DONE{Falta el mensaje de formato incorrecto}}{05/11}
% 			\RCitem{PC14}{\DONE{En la trayectoria alternativa C el campo no debería de indicarse}}{05/11}
% 			\RCitem{PC15}{\DONE{En el paso 8 indicar que se trata de una pantalla en lugar de decir ``figura''}}{05/11}
% 			\RCitem{PC16}{\DONE{¿Se hablará de cómo cerrar la sesión?, en el sistema pasado se documento ``cerrar sesón'' como parte de la trayectoria principal del caso de uso ``Ingresar al sistema'' }}{05/11}
% 			\RCitem{PC17}{\DONE{En el segunto punto de extensión dice ``insripción''}}{05/11}
% 			
% 		\end{UClist}		
% 	}
%--------------------------------------------------------
	\UCccsection{Revisión Versión 0.3} % Anote la versión que se revisó.
	% FECHA: Anote la fecha en que se terminó la revisión.
	\UCccitem{Fecha}{11-11-14} 
	% EVALUADOR: Coloque el nombre completo de quien realizó la revisión.
	\UCccitem{Evaluador}{Natalia Giselle Hernández Sánchez}
	% RESULTADO: Coloque la palabra que mas se apegue al tipo de acción que el analista debe realizar.
	\UCccitem{Resultado}{Corregir}
	% OBSERVACIONES: Liste los cambios que debe realizar el Analista.
	\UCccitem{Observaciones}{
		\begin{UClist}
			% PC: Petición de Cambio, describa el trabajo a realizar, si es posible indique la causa de la PC. Opcionalmente especifique la fecha en que considera razonable que se deba terminar la PC. No olvide que la numeración no se debe reiniciar en una segunda o tercera revisión.
			\RCitem{PC1}{\DONE{Agregar a precondiciones el estado de la cuenta}}{Fecha de entrega}
			\RCitem{PC2}{\DONE{Agregar el paso de la trayectoria de validación del estado de la cuenta}}{Fecha de entrega}
			\RCitem{PC3}{\DONE{Agregar el mensaje de cuenta no activada a la sección de errores}}{Fecha de entrega}
			\RCitem{PC4}{\DONE{Verificar las ligas a los estados}}{Fecha de entrega}
			
		\end{UClist}		
	}
%--------------------------------------------------------

	\UCsection{Atributos}
	\UCitem{Actor}{\textbf{Paciente}}
	\UCitem{Propósito}{Agregar un Doctor.}
	\UCitem{Entradas}{
        \begin{UClist} 
           \UCli \textbf{Nombre}: \ioEscribir.
           \UCli \textbf{Apellido Paterno}: \ioEscribir.
           \UCli \textbf{Apellido Materno}: \ioEscribir.
           \UCli \textbf{Numero de Cédula profesional}: \ioEscribir
           \UCli \textbf{Correo electrónico}: \ioEscribir
        \end{UClist}}
	\UCitem{Salidas}{Ninguna.}
	\UCitem{Precondiciones}{
		Ninguna.
		}
	\UCitem{Postcondiciones}{
	    {\bf Interna:} El doctor podrá ser asociado a un tratamiento.	
	}
    \UCitem{Reglas de negocio}{

            \cdtIdRef{RN1}{Información correcta}: Verifica que la información introducida sea correcta.
            \cdtIdRef{RNX}{Doctor existente}: Verifica que el doctor que se quiere agregar ya exista en la aplicación.
    }
	\UCitem{Errores}{
	    \begin{UClist}
	    \UCli \cdtIdRef{MSG1}{Operación Exitosa}: Se muestra en la pantalla \textbf{Iniciar sesión} cuando se inicio la sesión de manera exitosa.	
		\UCli \cdtIdRef{MSG1}{Falta un dato requerido para efectuar la operación solicitada}: Se muestra en la pantalla \textbf{Iniciar sesión} cuando el actor omitió un dato marcado como requerido.
		\UCli \cdtIdRef{MSG2}{correo electrónico y/o contraseña incorrecto}: Se muestra en la pantalla \textbf{Iniciar sesión} indicando que el correo electrónico y/o contraseña son incorrectos.
		
	    \end{UClist}
	}
	\UCitem{Tipo}{Primario.}
	\UCitem{Fuente}{
	}
 \end{UseCase}

 \begin{UCtrayectoria}
 	
 	\UCpaso [\UCactor] Presiona el icono \textbf{Agregar Doctor} de la pantalla \textbf{Gestion Doctores}\refTray{A}
 	\UCpaso [\UCactor] Ingresa el Numero de Cedula Profesional. \label{doctor}
 	\UCpaso [\UCactor] Presiona el botón \cdtButton{Agregar nuevo doctor}
 	\UCpaso Muestra la cedula profesional ingresada en el paso \ref{doctor}.
 	\UCpaso [\UCactor] Ingresa el nombre del doctor.
 	\UCpaso [\UCactor] Ingresa el apellido paterno.
 	\UCpaso [\UCactor] Ingresa el apellido materno.
 	\UCpaso [\UCactor] Presiona el boton \cdtButton{Agregar nuevo doctor}.
 	\UCpaso [\UCsist] Verifica que los datos ingresados por el usuario sean correctos como lo indica la regla de negocio \cdtIdRef{RN-S1}{Información correcta}. \refTray{B}
 	\UCpaso [\UCsist] Verifica que no exista el doctor que se quiere agregar como lo indica la regla de negocio \cdtIdRef{RN-SX}{Doctor existente}. \refTray{C}
 	
 	
 	

 	
    
 \end{UCtrayectoria}

 \begin{UCtrayectoriaA}{A}{Se quiere agregar el doctor desde la pantalla \textbf{Agregar Nuevo Tratamiento.}}
	\UCpaso [\UCactor] Presiona el icono \textbf{Agregar Doctor} de la pantalla \textbf{Agregar nuevo tratamiento}
	\UCpaso [\UCactor] Ingresa el Numero de Cedula Profesional. \label{doctor}
	\UCpaso [\UCactor] Presiona el botón \cdtButton{Agregar nuevo doctor}
	\UCpaso Muestra la cedula profesional ingresada en el paso \ref{doctor}.
	\UCpaso [\UCactor] Ingresa el nombre del doctor.
	\UCpaso [\UCactor] Ingresa el apellido paterno.
	\UCpaso [\UCactor] Ingresa el apellido materno.
	\UCpaso [\UCactor] Presiona el boton \cdtButton{Agregar nuevo doctor}.
	\UCpaso [\UCsist] Verifica que los datos ingresados por el usuario sean correctos como lo indica la regla de negocio \cdtIdRef{RN-S1}{Información correcta}. \refTray{B}
	\UCpaso [\UCsist] Verifica que no exista el doctor que se quiere agregar como lo indica la regla de negocio \cdtIdRef{RN-SX}{Doctor existente}. \refTray{C}
	\UCpaso Extiende al caso de uso \cdtIdRef{CUR9}{Agregar Tratamiento}
\end{UCtrayectoriaA}


 \begin{UCtrayectoriaA}{B}{El actor no ingresó alguno de los datos requeridos.}
    \UCpaso[\UCsist] Muestra el mensaje \cdtIdRef{MSG2}{Falta un dato requerido para efectuar la operación solicitada} en la pantalla \textbf{}{Iniciar sesión}
   \UCpaso[] Continúa en el paso \ref{cur1:Acciones} de la trayectoria principal.
 \end{UCtrayectoriaA}
 
 \begin{UCtrayectoriaA}{C}{El doctor que se quiere registrar ya existe.}
 	
 	
    \UCpaso[\UCsist] Muestra el mensaje \cdtIdRef{MSGX}{Doctor existente} en la pantalla \textbf{Agregar Doctor}.
    \UCpaso Continúa con el paso \ref{doctor}.
 \end{UCtrayectoriaA}


 
\subsection{Puntos de extensión}


 

