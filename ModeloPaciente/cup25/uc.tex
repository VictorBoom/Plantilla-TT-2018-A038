\begin{UseCase}{CUR 25}{Cambiar Contraseña}
    {
    	Es importante estár cambiando regularmente una contraseña para mantener la seguridad de nuestros datos. Es por eso que este caso de uso permite al \textbf{Paciente} cambiar su contraseña cada que éste lo solicite.
    }
    \UCitem{Versión}{1.0}
    \UCccsection{Paciente}
    \UCitem{Autor}{Daniel Josue Fuentes Hernández}
    \UCccitem{Evaluador}{Victor Arquimedes Estrada Machuca}
    \UCitem{Operación}{Cambió de contraseña}
    \UCccitem{Prioridad}{Baja}
    \UCccitem{Complejidad}{Baja}
    \UCccitem{Volatilidad}{Baja}
    \UCccitem{Madurez}{Alta}
    \UCitem{Estatus}{Terminado}
    \UCitem{Fecha del último estatus}{05 de Octubre del 2018}

%% Copie y pegue este bloque tantas veces como revisiones tenga el caso de uso.
%% Esta sección la debe llenar solo el Revisor
% %--------------------------------------------------------
% 	\UCccsection{Revisión Versión XX} % Anote la versión que se revisó.
% 	% FECHA: Anote la fecha en que se terminó la revisión.
% 	\UCccitem{Fecha}{Fecha en que se termino la revisión} 
% 	% EVALUADOR: Coloque el nombre completo de quien realizó la revisión.
% 	\UCccitem{Evaluador}{Nombre de quien revisó}
% 	% RESULTADO: Coloque la palabra que mas se apegue al tipo de acción que el analista debe realizar.
% 	\UCccitem{Resultado}{Corregir, Desechar, Rehacer todo, terminar.}
% 	% OBSERVACIONES: Liste los cambios que debe realizar el Analista.
% 	\UCccitem{Observaciones}{
% 		\begin{UClist}
% 			% PC: Petición de Cambio, describa el trabajo a realizar, si es posible indique la causa de la PC. Opcionalmente especifique la fecha en que considera razonable que se deba terminar la PC. No olvide que la numeración no se debe reiniciar en una segunda o tercera revisión.
% 			\RCitem{PC1}{\TODO{Descripción del pendiente}}{Fecha de entrega}
% 			\RCitem{PC2}{\TODO{Descripción del pendiente}}{Fecha de entrega}
% 			\RCitem{PC3}{\TODO{Descripción del pendiente}}{Fecha de entrega}
% 		\end{UClist}		
% 	}
% %--------------------------------------------------------
% %--------------------------------------------------------
% 	\UCccsection{Revisión Versión 0.3} % Anote la versión que se revisó.
% 	% FECHA: Anote la fecha en que se terminó la revisión.
% 	\UCccitem{Fecha}{04/11/2014} 
% 	% EVALUADOR: Coloque el nombre completo de quien realizó la revisión.
% 	\UCccitem{Evaluador}{Natalia Giselle Hernández Sánchez}
% 	% RESULTADO: Coloque la palabra que mas se apegue al tipo de acción que el analista debe realizar.
% 	\UCccitem{Resultado}{Corregir}
% 	% OBSERVACIONES: Liste los cambios que debe realizar el Analista.
% 	\UCccitem{Observaciones}{
% 		\begin{UClist}
% 			% PC: Petición de Cambio, describa el trabajo a realizar, si es posible indique la causa de la PC. Opcionalmente especifique la fecha en que considera razonable que se deba terminar la PC. No olvide que la numeración no se debe reiniciar en una segunda o tercera revisión.
% 			\RCitem{PC2}{\DONE{Cambiar en la sección de entradas el nombre del atributo ``Nombre de usuario'' y la redirección}}{04/11}
% 			\RCitem{PC3}{\DONE{La liga de ``Contraseña'' está rota}}{04/11}
% 			\RCitem{PC4}{\DONE{Hay oraciones que no tienen punto al final, como las precondiciones}}{04/11}
% 			\RCitem{PC5}{\DONE{Quitar la palabra ``interna'' de las viñetasde los menús en la sección de postcondiciones}}{04/11}
% 			\RCitem{PC6}{\DONE{En las postcondiciones dice ``perfl''}}{04/11}
% 			\RCitem{PC7}{\DONE{Falta punto en el primer paso de la TP}}{05/11}
% 			\RCitem{PC8}{\DONE{El paso 6 de la trayectoria principal no debería de estar debido a que el nombre de usuario no siempre es la cct}}{05/11}
% 			\RCitem{PC9}{\DONE{El paso 5 de la TP podría ser general como ``Verifica que los datos ingresados sean correctos según la regla de negocio...'' para que se verifique la contraseña, en todos los campos se verifica la longitud}}{05/11}
% 			\RCitem{PC10}{\DONE{Hace falta mencionar el mensaje de longitud}}{05/11}
% 			\RCitem{PC11}{\DONE{Falta mencionar el mensaje 22 en la sección de errores}}{05/11}
% 			\RCitem{PC12}{\DONE{Falta el mensaje de que se ha excedido la longitud}}{05/11}
% 			\RCitem{PC13}{\DONE{Falta el mensaje de formato incorrecto}}{05/11}
% 			\RCitem{PC14}{\DONE{En la trayectoria alternativa C el campo no debería de indicarse}}{05/11}
% 			\RCitem{PC15}{\DONE{En el paso 8 indicar que se trata de una pantalla en lugar de decir ``figura''}}{05/11}
% 			\RCitem{PC16}{\DONE{¿Se hablará de cómo cerrar la sesión?, en el sistema pasado se documento ``cerrar sesón'' como parte de la trayectoria principal del caso de uso ``Ingresar al sistema'' }}{05/11}
% 			\RCitem{PC17}{\DONE{En el segunto punto de extensión dice ``insripción''}}{05/11}
% 			
% 		\end{UClist}		
% 	}
%--------------------------------------------------------
	\UCccsection{Revisión Versión 0.3} % Anote la versión que se revisó.
	% FECHA: Anote la fecha en que se terminó la revisión.
	\UCccitem{Fecha}{11-11-14} 
	% EVALUADOR: Coloque el nombre completo de quien realizó la revisión.
	\UCccitem{Evaluador}{Natalia Giselle Hernández Sánchez}
	% RESULTADO: Coloque la palabra que mas se apegue al tipo de acción que el analista debe realizar.
	\UCccitem{Resultado}{Corregir}
	% OBSERVACIONES: Liste los cambios que debe realizar el Analista.
	\UCccitem{Observaciones}{
		\begin{UClist}
			% PC: Petición de Cambio, describa el trabajo a realizar, si es posible indique la causa de la PC. Opcionalmente especifique la fecha en que considera razonable que se deba terminar la PC. No olvide que la numeración no se debe reiniciar en una segunda o tercera revisión.
			\RCitem{PC1}{\DONE{Agregar a precondiciones el estado de la cuenta}}{Fecha de entrega}
			\RCitem{PC2}{\DONE{Agregar el paso de la trayectoria de validación del estado de la cuenta}}{Fecha de entrega}
			\RCitem{PC3}{\DONE{Agregar el mensaje de cuenta no activada a la sección de errores}}{Fecha de entrega}
			\RCitem{PC4}{\DONE{Verificar las ligas a los estados}}{Fecha de entrega}
			
		\end{UClist}		
	}
%--------------------------------------------------------

	\UCsection{Atributos}
	\UCitem{Actor}{\textbf{Paciente}}
	\UCitem{Propósito}{Cambiar la contraseña.}
	\UCitem{Entradas}{
        \begin{UClist} 
           \UCli \textbf{Contraseña anterior}: \ioEscribir.
           \UCli \textbf{Contraseña nueva}: \ioEscribir.
           \UCli \textbf{Confirmación de contraseña}: \ioEscribir.
        \end{UClist}}
	\UCitem{Salidas}{\begin{UClist} 
			\UCli Contraseña.
			\UCli \cdtIdRef{MSG1}{Operación Exitosa}: Se muestra en la pantalla \textbf{Datos personales} cuando se cambió la contraseña de manera exitosa.
	\end{UClist}}
	\UCitem{Precondiciones}{Ninguna}
	\UCitem{Postcondiciones}{
	    {\bf Interna:} Cambiará la contraseña del usuario.	
	}
    \UCitem{Reglas de negocio}{

            \cdtIdRef{RN}{Datos Requeridos}:Verifica que todos los datos requeridos sean proporcionados.
            
            \cdtIdRef{RN}{Formato de Contraseña Incorrecto}:Verifica que el correo electrónico cumpla con el formato correcto.
            
            \cdtIdRef{RN}{Confirmación de Contraseña Incorrecta}: Verifica que la contraseña coincida con la confirmación de contraseña:
    }
	\UCitem{Errores}{
	    \begin{UClist}
	    \UCli \cdtIdRef{MSG1}{Falta un dato requerido para efectuar la operación solicitada}: Se muestra en la pantalla \textbf{Datos Personales} cuando el actor omitió un dato marcado como requerido.
		\UCli \cdtIdRef{MSG2}{Correo electrónico y/o contraseña incorrecto}: Se muestra en la pantalla \textbf{Datos Personales} indicando que el correo electrónico y/o contraseña son incorrectos.
		\UCli \cdtIdRef {MSG}{Validación de contraseña incorrecta}: Se muestra en la pantalla \textbf{Datos Personales} indicando que la confirmación de contraseña no coincide con la contraseña ingresada.
		
	    \end{UClist}
	}
	\UCitem{Tipo}{Primario.}
	\UCitem{Fuente}{
	}
 \end{UseCase}

 \begin{UCtrayectoria}
 	
 	\UCpaso [\UCactor] Presiona el botón \textbf{Cambiar Contraseña} de la pantalla \textbf{Editar Datos Personales}.
 	\UCpaso Muestra la pantalla \textbf{Cambiar Contraseña}.
 	\UCpaso [\UCactor] Ingresa la Contraseña Anterior .
 	\UCpaso [\UCactor] Ingresa la Contraseña Nueva 
 	\UCpaso [\UCactor]Ingresa la Confirmación de Contraseña.
 	\UCpaso [\UCactor] Presiona el botón \textbf{Cambiar contraseña} en la pantalla \textbf{Cambiar Contraseña}.
 	\UCpaso Verifica que todos los datos requeridos sean proporcionados con base a la regla de negocio \cdtIdRef{RN}{Datos Requeridos}.
 	\UCpaso Verifica que las contraseñas cumplan con el formato correcto con base en la regla de negocio \cdtIdRef{RN}{Formato de Contraseña Incorrecto}.
 	\UCpaso Verifica que la contraseña nueva coincida con la confirmación de la contraseña con base a la regla de negocio \cdtIdRef{RN}{Confirmación de Contraseña Incorrecta}
 	\UCpaso Cambia la contraseña del usuario.
 	\UCpaso Muestra el mensaje \cdtIdRef{MSG1}{Operación Exitosa} en la pantalla \textbf{Editar Datos Personales}.
     
 \end{UCtrayectoria}

 \begin{UCtrayectoriaA}{A}{El actor desea cancelar la operación.}
 	\UCpaso [\UCactor] Da clic en el botón \textbf{Cancelar} de la pantalla \textbf{Cambiar Contraseña}.
 	\UCpaso Muestra la pantalla \textbf{Editar Datos Personales}.
    
 \end{UCtrayectoriaA}

 \begin{UCtrayectoriaA}{B}{El actor no ingresó alguno de los datos requeridos.}
	\UCpaso [\UCactor] Muestra el mensaje \cdtIdRef{MSG1}{Falta un dato requerido para efectuar la operación solicitada} indicando que falta un dato que es obligatorio.
	\UCpaso Regresa al paso 3 de la Trayectoria Principal.
	
\end{UCtrayectoriaA} 

\begin{UCtrayectoriaA}{C}{El actor ingresó una contraseña con un formato incorrecto}
	\UCpaso [\UCactor] Muestra el mensaje \cdtIdRef{MSG2}{Correo electrónico y/o contraseña incorrecto} indicando que la contraseña ingresada es incorrecta.
	\UCpaso Regresa al paso 3 de la Trayectoria Principal.
	
\end{UCtrayectoriaA} 


\begin{UCtrayectoriaA}{D}{El actor ingresó una confirmación de contraseña que no coincide con la contraseña nueva.}
	\UCpaso [\UCactor] Muestra el mensaje  \cdtIdRef {MSG}{Validación de contraseña incorrecta} indicando que la confirmación de contraseña ingresada no coincide con la nueva contraseña.
	\UCpaso Regresa al paso 4 de la Trayectoria Principal.
	
\end{UCtrayectoriaA} 

