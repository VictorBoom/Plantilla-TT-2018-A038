\subsection{IUP1 Iniciar sesión}
 
\subsubsection{Objetivo}

    Esta pantalla permite al actor \textbf{Paciente} iniciar sesión en la aplicación, recuperar la contraseña de acceso a la aplicación o registrarse para formar parte del mismo.

\subsubsection{Diseño}

    En la figura \ref{IUP1} se muestra la pantalla ``Iniciar sesión'', por medio de la cual se podrá acceder a la aplicación. \\

	\IUfig[.3]{pantallas/iniciarsesion}{IUP1}{Iniciar sesión}

\subsubsection{Comandos}
\begin{itemize}
    \item \cdtButton{Iniciar Sesión}: Permite al actor ingresar al sistema, dirige a la pantalla \textbf{Menú Principal} que se muestra en la figura %\ref{fig:menúPrincipal}.
    \item \cdtButton{¿Olvidaste tu contraseña?}: Permite al actor solicitar el envío de su contraseña a su correo, dirige a la pantalla \cdtIdRef{IUP2}{Recuperar Contraseña}.
    \item \cdtButton{Crear Cuenta}: Se utiliza para crear una nueva cuenta para tener acceso a la aplicación, dirige a la pantalla \cdtIdRef{IUP3}{Registro de Cuenta}.
\end{itemize}

\subsubsection{Mensajes}

\begin{description}
    \item[\cdtIdRef{MSG2}{Falta un dato requerido para efectuar la operación solicitada}:] Se muestra en la pantalla \cdtIdRef{IUP1}{Iniciar de sesión} cuando el actor omitió un dato marcado como requerido.
    \item[\cdtIdRef{MSG3}{Nombre de usuario y/o contraseña incorrecto}:] Se muestra en la pantalla \cdtIdRef{IUP1}{Iniciar de sesión} indicando que el nombre de usuario y/o contraseña son incorrectos.
\end{description}
