\begin{UseCase}{CUR 26}{Consultar Recordatorios}
    {
    	Un paciente podrá consultar los recordatorios con los que cuente en ese momento, de esta manera podrá saber cuales son las medicinas mas próximas a tomar, saber su nombre y la dosis.
    }
    \UCitem{Versión}{1.0}
    \UCccsection{Paciente}
    \UCitem{Autor}{Victor Arquimedes Estrada Machuca}
    \UCccitem{Evaluador}{Daniel Josue Fuentes Hernández}
    \UCitem{Operación}{Consultar recordatorios}
    \UCccitem{Prioridad}{Media}
    \UCccitem{Complejidad}{Baja}
    \UCccitem{Volatilidad}{Baja}
    \UCccitem{Madurez}{Alta}
    \UCitem{Estatus}{Terminado}
    \UCitem{Fecha del último estatus}{05 de Octubre del 2018}

%% Copie y pegue este bloque tantas veces como revisiones tenga el caso de uso.
%% Esta sección la debe llenar solo el Revisor
% %--------------------------------------------------------
% 	\UCccsection{Revisión Versión XX} % Anote la versión que se revisó.
% 	% FECHA: Anote la fecha en que se terminó la revisión.
% 	\UCccitem{Fecha}{Fecha en que se termino la revisión} 
% 	% EVALUADOR: Coloque el nombre completo de quien realizó la revisión.
% 	\UCccitem{Evaluador}{Nombre de quien revisó}
% 	% RESULTADO: Coloque la palabra que mas se apegue al tipo de acción que el analista debe realizar.
% 	\UCccitem{Resultado}{Corregir, Desechar, Rehacer todo, terminar.}
% 	% OBSERVACIONES: Liste los cambios que debe realizar el Analista.
% 	\UCccitem{Observaciones}{
% 		\begin{UClist}
% 			% PC: Petición de Cambio, describa el trabajo a realizar, si es posible indique la causa de la PC. Opcionalmente especifique la fecha en que considera razonable que se deba terminar la PC. No olvide que la numeración no se debe reiniciar en una segunda o tercera revisión.
% 			\RCitem{PC1}{\TODO{Descripción del pendiente}}{Fecha de entrega}
% 			\RCitem{PC2}{\TODO{Descripción del pendiente}}{Fecha de entrega}
% 			\RCitem{PC3}{\TODO{Descripción del pendiente}}{Fecha de entrega}
% 		\end{UClist}		
% 	}
% %--------------------------------------------------------
% %--------------------------------------------------------
% 	\UCccsection{Revisión Versión 0.3} % Anote la versión que se revisó.
% 	% FECHA: Anote la fecha en que se terminó la revisión.
% 	\UCccitem{Fecha}{04/11/2014} 
% 	% EVALUADOR: Coloque el nombre completo de quien realizó la revisión.
% 	\UCccitem{Evaluador}{Natalia Giselle Hernández Sánchez}
% 	% RESULTADO: Coloque la palabra que mas se apegue al tipo de acción que el analista debe realizar.
% 	\UCccitem{Resultado}{Corregir}
% 	% OBSERVACIONES: Liste los cambios que debe realizar el Analista.
% 	\UCccitem{Observaciones}{
% 		\begin{UClist}
% 			% PC: Petición de Cambio, describa el trabajo a realizar, si es posible indique la causa de la PC. Opcionalmente especifique la fecha en que considera razonable que se deba terminar la PC. No olvide que la numeración no se debe reiniciar en una segunda o tercera revisión.
% 			\RCitem{PC2}{\DONE{Cambiar en la sección de entradas el nombre del atributo ``Nombre de usuario'' y la redirección}}{04/11}
% 			\RCitem{PC3}{\DONE{La liga de ``Contraseña'' está rota}}{04/11}
% 			\RCitem{PC4}{\DONE{Hay oraciones que no tienen punto al final, como las precondiciones}}{04/11}
% 			\RCitem{PC5}{\DONE{Quitar la palabra ``interna'' de las viñetasde los menús en la sección de postcondiciones}}{04/11}
% 			\RCitem{PC6}{\DONE{En las postcondiciones dice ``perfl''}}{04/11}
% 			\RCitem{PC7}{\DONE{Falta punto en el primer paso de la TP}}{05/11}
% 			\RCitem{PC8}{\DONE{El paso 6 de la trayectoria principal no debería de estar debido a que el nombre de usuario no siempre es la cct}}{05/11}
% 			\RCitem{PC9}{\DONE{El paso 5 de la TP podría ser general como ``Verifica que los datos ingresados sean correctos según la regla de negocio...'' para que se verifique la contraseña, en todos los campos se verifica la longitud}}{05/11}
% 			\RCitem{PC10}{\DONE{Hace falta mencionar el mensaje de longitud}}{05/11}
% 			\RCitem{PC11}{\DONE{Falta mencionar el mensaje 22 en la sección de errores}}{05/11}
% 			\RCitem{PC12}{\DONE{Falta el mensaje de que se ha excedido la longitud}}{05/11}
% 			\RCitem{PC13}{\DONE{Falta el mensaje de formato incorrecto}}{05/11}
% 			\RCitem{PC14}{\DONE{En la trayectoria alternativa C el campo no debería de indicarse}}{05/11}
% 			\RCitem{PC15}{\DONE{En el paso 8 indicar que se trata de una pantalla en lugar de decir ``figura''}}{05/11}
% 			\RCitem{PC16}{\DONE{¿Se hablará de cómo cerrar la sesión?, en el sistema pasado se documento ``cerrar sesón'' como parte de la trayectoria principal del caso de uso ``Ingresar al sistema'' }}{05/11}
% 			\RCitem{PC17}{\DONE{En el segunto punto de extensión dice ``insripción''}}{05/11}
% 			
% 		\end{UClist}		
% 	}
%--------------------------------------------------------
	\UCccsection{Revisión Versión 0.3} % Anote la versión que se revisó.
	% FECHA: Anote la fecha en que se terminó la revisión.
	\UCccitem{Fecha}{11-11-14} 
	% EVALUADOR: Coloque el nombre completo de quien realizó la revisión.
	\UCccitem{Evaluador}{Natalia Giselle Hernández Sánchez}
	% RESULTADO: Coloque la palabra que mas se apegue al tipo de acción que el analista debe realizar.
	\UCccitem{Resultado}{Corregir}
	% OBSERVACIONES: Liste los cambios que debe realizar el Analista.
	\UCccitem{Observaciones}{
		\begin{UClist}
			% PC: Petición de Cambio, describa el trabajo a realizar, si es posible indique la causa de la PC. Opcionalmente especifique la fecha en que considera razonable que se deba terminar la PC. No olvide que la numeración no se debe reiniciar en una segunda o tercera revisión.
			\RCitem{PC1}{\DONE{Agregar a precondiciones el estado de la cuenta}}{Fecha de entrega}
			\RCitem{PC2}{\DONE{Agregar el paso de la trayectoria de validación del estado de la cuenta}}{Fecha de entrega}
			\RCitem{PC3}{\DONE{Agregar el mensaje de cuenta no activada a la sección de errores}}{Fecha de entrega}
			\RCitem{PC4}{\DONE{Verificar las ligas a los estados}}{Fecha de entrega}
			
		\end{UClist}		
	}
%--------------------------------------------------------

	\UCsection{Atributos}
	\UCitem{Actor}{\textbf{Paciente}}
	\UCitem{Propósito}{Cambiar la contraseña.}
	\UCitem{Entradas}{
        Ninguna
    }
	\UCitem{Salidas}{
		\begin{UClist} 
			\UCli Nombre del medicamento
			\UCli Tratamiento al que pertenece el medicamento.
			\UCli Numero de píldoras restantes.
			\UCli Fecha de próxima toma
		\end{UClist}
}
	\UCitem{Precondiciones}{Ninguna}
	\UCitem{Postcondiciones}{
	    Ninguna.	
	}
    \UCitem{Reglas de negocio}{

            Ninguna.
    }
	\UCitem{Errores}{
	   
	     \cdtIdRef{MSGX}{No se pudieron obtener los datos}: Se muestra en la pantalla \textbf{Gestión notificaciones} cuando el actor omitió un dato marcado como requerido.
		
	}
	\UCitem{Tipo}{Primario.}
	\UCitem{Fuente}{
	}
 \end{UseCase}

 \begin{UCtrayectoria}
 	
 	\UCpaso [\UCactor] Presiona el icono \textbf{Consultar Recordatorios} en la pantalla \textbf{Gestión recordatorios}
 	\UCpaso Verifica que haya recordatorios. \refTray{A}
 	\UCpaso Obtiene el nombre del medicamento.
 	\UCpaso Obtiene el tratamiento al que pertenece el tratamiento.
 	\UCpaso Obtiene el numero de píldoras restantes.
 	\UCpaso Obtiene el horario del recordatorio.
 	\UCpaso Ordena los recordatorios según el horario del recordatorio.
 	\UCpaso Muestra la pantalla \textbf{Consultar recordatorios}
 	
     
 \end{UCtrayectoria}

 \begin{UCtrayectoriaA}{A}{El actor no tiene recordatorios.}
 	\UCpaso Muestra el mensaje \cdtIdRef{MSGX}{No hay recordatorios}.
 	\UCpaso Muestra la pantalla \textbf{Gestión de Recordatorios}.
    
 \end{UCtrayectoriaA}



