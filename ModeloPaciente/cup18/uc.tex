\begin{UseCase}{CUR 18}{Editar Auxiliar}
    {
    	Este caso de uso permite al \textbf{Paciente} editar los datos de un Auxiliar en caso de que haya registrado un dato incorrecto o desee cambiar su teléfono o correo electrónico.
    }
    \UCitem{Versión}{1.0}
    \UCccsection{Paciente}
    \UCitem{Autor}{Daniel Josue Fuentes Hernández}
    \UCccitem{Evaluador}{Victor Arquimedes Estrada Machuca}
    \UCitem{Operación}{Editar Auxiliar}
    \UCccitem{Prioridad}{Alta}
    \UCccitem{Complejidad}{Baja}
    \UCccitem{Volatilidad}{Baja}
    \UCccitem{Madurez}{Alta}
    \UCitem{Estatus}{Terminado}
    \UCitem{Fecha del último estatus}{05 de Octubre del 2018}

%% Copie y pegue este bloque tantas veces como revisiones tenga el caso de uso.
%% Esta sección la debe llenar solo el Revisor
% %--------------------------------------------------------
% 	\UCccsection{Revisión Versión XX} % Anote la versión que se revisó.
% 	% FECHA: Anote la fecha en que se terminó la revisión.
% 	\UCccitem{Fecha}{Fecha en que se termino la revisión} 
% 	% EVALUADOR: Coloque el nombre completo de quien realizó la revisión.
% 	\UCccitem{Evaluador}{Nombre de quien revisó}
% 	% RESULTADO: Coloque la palabra que mas se apegue al tipo de acción que el analista debe realizar.
% 	\UCccitem{Resultado}{Corregir, Desechar, Rehacer todo, terminar.}
% 	% OBSERVACIONES: Liste los cambios que debe realizar el Analista.
% 	\UCccitem{Observaciones}{
% 		\begin{UClist}
% 			% PC: Petición de Cambio, describa el trabajo a realizar, si es posible indique la causa de la PC. Opcionalmente especifique la fecha en que considera razonable que se deba terminar la PC. No olvide que la numeración no se debe reiniciar en una segunda o tercera revisión.
% 			\RCitem{PC1}{\TODO{Descripción del pendiente}}{Fecha de entrega}
% 			\RCitem{PC2}{\TODO{Descripción del pendiente}}{Fecha de entrega}
% 			\RCitem{PC3}{\TODO{Descripción del pendiente}}{Fecha de entrega}
% 		\end{UClist}		
% 	}
% %--------------------------------------------------------
% %--------------------------------------------------------
% 	\UCccsection{Revisión Versión 0.3} % Anote la versión que se revisó.
% 	% FECHA: Anote la fecha en que se terminó la revisión.
% 	\UCccitem{Fecha}{04/11/2014} 
% 	% EVALUADOR: Coloque el nombre completo de quien realizó la revisión.
% 	\UCccitem{Evaluador}{Natalia Giselle Hernández Sánchez}
% 	% RESULTADO: Coloque la palabra que mas se apegue al tipo de acción que el analista debe realizar.
% 	\UCccitem{Resultado}{Corregir}
% 	% OBSERVACIONES: Liste los cambios que debe realizar el Analista.
% 	\UCccitem{Observaciones}{
% 		\begin{UClist}
% 			% PC: Petición de Cambio, describa el trabajo a realizar, si es posible indique la causa de la PC. Opcionalmente especifique la fecha en que considera razonable que se deba terminar la PC. No olvide que la numeración no se debe reiniciar en una segunda o tercera revisión.
% 			\RCitem{PC2}{\DONE{Cambiar en la sección de entradas el nombre del atributo ``Nombre de usuario'' y la redirección}}{04/11}
% 			\RCitem{PC3}{\DONE{La liga de ``Contraseña'' está rota}}{04/11}
% 			\RCitem{PC4}{\DONE{Hay oraciones que no tienen punto al final, como las precondiciones}}{04/11}
% 			\RCitem{PC5}{\DONE{Quitar la palabra ``interna'' de las viñetasde los menús en la sección de postcondiciones}}{04/11}
% 			\RCitem{PC6}{\DONE{En las postcondiciones dice ``perfl''}}{04/11}
% 			\RCitem{PC7}{\DONE{Falta punto en el primer paso de la TP}}{05/11}
% 			\RCitem{PC8}{\DONE{El paso 6 de la trayectoria principal no debería de estar debido a que el nombre de usuario no siempre es la cct}}{05/11}
% 			\RCitem{PC9}{\DONE{El paso 5 de la TP podría ser general como ``Verifica que los datos ingresados sean correctos según la regla de negocio...'' para que se verifique la contraseña, en todos los campos se verifica la longitud}}{05/11}
% 			\RCitem{PC10}{\DONE{Hace falta mencionar el mensaje de longitud}}{05/11}
% 			\RCitem{PC11}{\DONE{Falta mencionar el mensaje 22 en la sección de errores}}{05/11}
% 			\RCitem{PC12}{\DONE{Falta el mensaje de que se ha excedido la longitud}}{05/11}
% 			\RCitem{PC13}{\DONE{Falta el mensaje de formato incorrecto}}{05/11}
% 			\RCitem{PC14}{\DONE{En la trayectoria alternativa C el campo no debería de indicarse}}{05/11}
% 			\RCitem{PC15}{\DONE{En el paso 8 indicar que se trata de una pantalla en lugar de decir ``figura''}}{05/11}
% 			\RCitem{PC16}{\DONE{¿Se hablará de cómo cerrar la sesión?, en el sistema pasado se documento ``cerrar sesón'' como parte de la trayectoria principal del caso de uso ``Ingresar al sistema'' }}{05/11}
% 			\RCitem{PC17}{\DONE{En el segunto punto de extensión dice ``insripción''}}{05/11}
% 			
% 		\end{UClist}		
% 	}
%--------------------------------------------------------
	\UCccsection{Revisión Versión 0.3} % Anote la versión que se revisó.
	% FECHA: Anote la fecha en que se terminó la revisión.
	\UCccitem{Fecha}{11-11-14} 
	% EVALUADOR: Coloque el nombre completo de quien realizó la revisión.
	\UCccitem{Evaluador}{Natalia Giselle Hernández Sánchez}
	% RESULTADO: Coloque la palabra que mas se apegue al tipo de acción que el analista debe realizar.
	\UCccitem{Resultado}{Corregir}
	% OBSERVACIONES: Liste los cambios que debe realizar el Analista.
	\UCccitem{Observaciones}{
		\begin{UClist}
			% PC: Petición de Cambio, describa el trabajo a realizar, si es posible indique la causa de la PC. Opcionalmente especifique la fecha en que considera razonable que se deba terminar la PC. No olvide que la numeración no se debe reiniciar en una segunda o tercera revisión.
			\RCitem{PC1}{\DONE{Agregar a precondiciones el estado de la cuenta}}{Fecha de entrega}
			\RCitem{PC2}{\DONE{Agregar el paso de la trayectoria de validación del estado de la cuenta}}{Fecha de entrega}
			\RCitem{PC3}{\DONE{Agregar el mensaje de cuenta no activada a la sección de errores}}{Fecha de entrega}
			\RCitem{PC4}{\DONE{Verificar las ligas a los estados}}{Fecha de entrega}
			
		\end{UClist}		
	}
%--------------------------------------------------------

	\UCsection{Atributos}
	\UCitem{Actor}{\textbf{Paciente}}
	\UCitem{Propósito}{Editar Auxiliar.}
	\UCitem{Entradas}{
        \begin{UClist} 
        	\UCli \textbf{Nombre}: \ioEscribir.
        	\UCli \textbf{Apellido Paterno}: \ioEscribir.
        	\UCli \textbf{Apellido Materno}: \ioEscribir.
        	\UCli \textbf{Teléfono}: \ioEscribir.
        	\UCli \textbf{Correo Electrónico}: \ioEscribir.
    \end{UClist}}
	\UCitem{Salidas}{\begin{UClist} 
			\UCli Nombre del auxiliar.
			\UCli Apellido paterno del auxiliar.
			\UCli Apellido materno del auxiliar.
			\UCli Teléfono del auxiliar.
			\UCli Correo electrónico del auxiliar.
			\UCli \cdtIdRef{MSG1}{Operación Exitosa}: Se muestra en la pantalla \textbf{Auxiliares} cuando se agregó el auxiliar de manera exitosa.
	\end{UClist}}
	\UCitem{Precondiciones}{Ninguna.}
	\UCitem{Postcondiciones}{
	     {\bf Interna:} Editará los datos del Auxiliar en la aplicación.	
	}
    \UCitem{Reglas de negocio}{

            \cdtIdRef{RN}{Datos Requeridos}:Verifica que todos los datos requeridos, sean proporcionados.
            
            \cdtIdRef{RN}{Formato de Correo Electrónico}:Verifica que el correo electrónico cumpla con el formato correcto.
    }
	\UCitem{Errores}{
	    \begin{UClist}
	    
	    \UCli \cdtIdRef{MSG1}{Falta un dato requerido para efectuar la operación solicitada}: Se muestra en la pantalla \textbf{Editar Auxiliar} cuando el actor omitió un dato marcado como requerido.
	    \UCli \cdtIdRef{MSG2}{correo electrónico y/o contraseña incorrecto}: Se muestra en la pantalla \textbf{Editar Auxiliar} indicando que el correo electrónico y/o contraseña son incorrectos.
		
	    \end{UClist}
	}
	\UCitem{Tipo}{Primario.}
	\UCitem{Fuente}{
	}
 \end{UseCase}

 \begin{UCtrayectoria}
 	
 	\UCpaso [\UCactor] Presiona el ícono \textbf{Editar Auxiliar} asociado al auxiliar que desea editar en la pantalla \textbf{Auxiliares}.
 	\UCpaso Muestra la pantalla \textbf{Editar Auxiliar}.
 	\UCpaso [\UCactor] Edita la información que deseas editar en la pantalla \textbf{Editar Auxiliar}.
 	\UCpaso [\UCactor] Presiona el botón \textbf{Aceptar}.
 	\UCpaso Verifica que no haga falta ningún dato requerido con base a la regla de negocio \textbf{RN Datos Requeridos}.
 	\UCpaso Verifica que el correo electrónico cumpla con el formato correcto con base a la regla de negocio \textbf{RN Formato de Correo Electrónico}.
 	\UCpaso Actualiza los datos del auxiliar que fueron editados.
 	\UCpaso Muestra el mensaje \textbf{MSG1 Operación Exitosa} en la pantalla \textbf{Auxiliares}.
 	   
 \end{UCtrayectoria}

 \begin{UCtrayectoriaA}{A}{El actor desea cancelar la operación.}
	\UCpaso[\UCactor] Solicita cancelar la operación oprimiendo el botón Cancelar en la pantalla \textbf{Editar Auxiliar}.
	\UCpaso Regresa a la pantalla \textbf{Auxiliares}.
\end{UCtrayectoriaA}

\begin{UCtrayectoriaA}{B}{El actor no ingresó un dato requerido.}
	\UCpaso Muestra el mensaje \cdtIdRef{MSG}{Falta un dato requerido para efectuar la operaciónn solicitada} en la pantalla \textbf{Editar Auxiliar}.
	\UCpaso Continúa con el paso 3 de la trayectoria principal.
\end{UCtrayectoriaA}

\begin{UCtrayectoriaA}{C}{El actor ingresó un correo electrónico y/o contraseña con un formato incorrecto.}
	\UCpaso Muestra el mensaje \cdtIdRef{MSG}{Correo electrónico y/o contraseña incorrecta} en la pantalla \textbf{Editar Auxiliar} indicando que el usuario y/o contraseña son incorrectos.
	\UCpaso Continúa con el paso 3 de la trayectoria principal.
\end{UCtrayectoriaA} 
